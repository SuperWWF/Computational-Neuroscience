\documentclass[12pt]{article}
\usepackage{amsfonts, epsfig}

\usepackage{graphicx}
\usepackage{fancyhdr}
\pagestyle{fancy}
\lfoot{\texttt{github.com/COMS30127}}
\lhead{Computation Neuroscience - Coursework 2 - spike trains}
\rhead{\thepage}
\cfoot{}
\begin{document}

\section*{Coursework 2}

This coursework relates to the properties of spike train. These
questions are adapted from the exercises given in Part 1.1 of\\
\\
\texttt{http://www.gatsby.ucl.ac.uk/$\sim$dayan/book/exercises.html}\\ 
\\
in their original form they are exercises for the book Dayan and Abbott,
a recommended text book for this course.

We give helper code for completing this coursework in Python and Julia, but feel free to use MATLAB or another language if you prefer.

The first question relates to Poisson processes, a Poisson process
is a memoryless random process producing a time series of events; in
the application to spike trains an event corresponds to a spike. By
memoryless it is meant that the chance of an event depends only on a
rate function $r(t)$ and not on the history of past events. The idea
is that the probability of an event in a small interval $\delta t$ wide is $r(t)\delta t$.

Imagine a person fishing in the sea, the chance of catching a fish
might depend on the time of day but it doesn't depend on how many
fishes the person has already caught, however, if they are fishing in
a small pond the chance of catching a fish would diminish as they
catch fish, so fishing in the sea is a Poisson process, but fishing in
a pond is not. In the two questions here the Poisson process is
homogenous, that is, the rate is constant. Of course, if there is a
refractory period, that is a time during which the neuron cannot
spike, the spike train is not a Poisson process, but in this question
a Poisson process is used to generate the spike train.

It can be proved, with a nice argument you are urged to look up, that
the distribution of inter-event intervals $t$, here inter-spike intervals,
is given by
\begin{equation}
p(t)=\frac{1}{r}e^{-rt}
\end{equation}
The code supplied in \texttt{poisson.py} and \texttt{poisson.jl}
generates intervals according to this distribution. A distinctive
feature of a Poisson process is that its Fano factor and coefficient
of variation. The aim here is to calculate these quantities for
simulated and real spike trains.

The Fano factor is defined as
\begin{equation}
F=\frac{\sigma^2}{\mu}
\end{equation}
where $\sigma^2$ is a variance and $\mu$ is an average.  In the
case of spike trains it is usually applied to the spike count; so to
calculate it you divide the spike train into intervals and work out
the spike count for each interval. The Fano factor is calculated using
the average and variance for these counts. The coefficient of variation is 
\begin{equation}
C_v=\frac{\sigma}{\mu}
\end{equation}
where $\sigma$ is a standard deviation and $\mu$ is an average. In the
case of spike trains this is usually applied to the inter-spike
interval, that is the time difference between successive spikes.

\subsection*{Question 1}

In the \texttt{spike\_trains} folder you will find
\texttt{poisson.jl} and \texttt{poisson.py}. These contain a function
which will generate spike trains simulated using a Poisson process
with a refractory period. Calculate the Fano factor of the $\textbf{spike count}$
and coefficient of variation of the $\textbf{inter-spike interval}$ for 1000
seconds of spike train with a firing rate of 35 Hz, both with no
refractory period and with a refractory period of 5 ms. In the case of
the Fano factor the count should be performed over windows of width 10
ms, 50 ms and 100 ms. [5 marks]

\subsection*{Question 2}

In the \texttt{spike\_trains} folder you will find the data file
\texttt{rho.dat}. This is contains data collected and provided by Rob
de Ruyter van Steveninck from a fly H1 neuron responding to an
approximate white-noise visual motion stimulus. Data were collected
for 20 minutes at a sampling rate of 500 Hz. In the file, rho is a
vector that gives the sequence of spiking events or non-events at the
sampled time, that is, every 2 ms. Simple programmes, \texttt{load.jl}
and \texttt{load.py}, are supplied to load this data file and the data
file described in the next question.

Calculate the Fano factor and coefficient of variation for this spike
train as for the simulated spike trains above. [5 marks]

\subsection*{Question 3}
For the same spike train given in the rho vector, calculate and plot the autocorrelogram over the range $-200$ ms to $+200$ ms. You can calculate this using the method outlined in the lecture, or using an existing python/julia package. You can plot using Python or Julia or you can export the data and plot using another package such as gnuplot. Do not upload a screen shot of the plot, the plot should be exported into whatever format you upload. [5 marks]

\subsection*{Question 4} 

In the \texttt{spike\_trains} folder you will find the data file
\texttt{stim.dat}. This give the motion stimulus that evoked the spike
train in \texttt{rho.dat}. Calculate and plot the spike triggered
average over a 100 ms window.  [5 marks]

\subsection*{COMSM2127}

Calculate the stimulus triggered by pairs of spikes, that is for
intervals of 2 ms, 10 ms, 20 ms and 50 ms calculate the average
stimulus before a pair of spikes seperated by that interval; do this
for both the case where the spikes are not neccesarily adjacent and
the case where they are. [5 marks]

\subsection*{Submission instructions}

Please submit your code along with a short document giving the answers
to Q1 and Q2 and the plots from Q3 and Q4 and for COMSM2127 students the extra
plot for that section. Include enough text to make it clear what you
are presenting, for example \lq{}the Fano factor for the Poisson train
with zero refractory period is $F=$\ldots\rq{}. Submit the document as
a pdf, not a doc. If (but only if) Blackboard will not let you upload separate files for the code and pdf, then please combine both files in a .zip.

The deadline is 1pm on 27th April 2020.

\end{document}
